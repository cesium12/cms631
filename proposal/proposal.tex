\documentclass[12pt]{article}

%% Include packages
%%\usepackage{times}
\usepackage[margin=1in,letterpaper,portrait]{geometry}
\usepackage{amsmath}
\usepackage{amssymb}
\usepackage{amsthm}
\usepackage{fancyhdr}
\usepackage[pdftex]{graphicx}
\usepackage{listings}
\usepackage{eucal}
\usepackage[
  pdftitle={},%
  pdfauthor={},%
  pdfsubject={},%
  pdfkeywords={},%
  pdfstartview=FitH,%
  bookmarks=true,%
  bookmarksopen=true,%
  breaklinks=true,%
  colorlinks=true,%
  linkcolor=black,anchorcolor=black,%
  citecolor=black,filecolor=black,%
  menucolor=black,pagecolor=black,%
  urlcolor=black,%
  pdftex]{hyperref}

%% Modify these variables
\newcommand{\student}{Alexander Chernyakhovsky and Alan Huang}
\newcommand{\studentemail}{achernya@mit.edu}
\newcommand{\course}{CMS.631}
\newcommand{\pset}{Problem Set \#8 -- Proposal}

%% DO NOT MODIFY THIS SECTION

\fancyhf{}
\lhead{\course \\ \pset}
\rhead{\student \\ \studentemail}
\cfoot{\thepage}
\addtolength{\headheight}{30pt}
\renewcommand{\headrulewidth}{0.4pt}
\renewcommand{\footrulewidth}{0.4pt}

\pagestyle{fancy}

\newenvironment{problemset}{\begin{enumerate}}{\end{enumerate}}
\newenvironment{problem}[1]{\item #1\\}{}

%% END DO NOT MODIFY THIS SECTION

%% PUT MACROS HERE
\newcommand{\bra}{\left\langle}
\newcommand{\ket}{\right\rangle}
%%

\begin{document}

% Set up document title
\title{\pset}
\author{\student}
\date{}
\maketitle
\thispagestyle{fancy}

\section{Theme}
The overall goal is to demonstrate the complexity behind the modern
consumer computer; to build an appreciation for inner workings of the
digital devices that we find are integral to our daily lives.

\section{Introduction}
The modern desktop/laptop computer is ubiquitous---yet, very few
people outside of the engineers that design them and the programmers
that build the software that runs them fully understand how a computer
functions.  Most explanations cover different facets of the computer:
logic circuits, von Neumann architecture, operating systems,
processes, threads, interaction between programs---and so on.
However, all of these approaches are suitable for an engineer, that
can then extrapolate how they will interact.

Such extrapolation is difficult, but provides a sense of scale to the
overall system, and an appreciation that only comes with
understanding.  We endevour to present a new visualization that
accurately demonstrates the various tasks a computer must perform to
run a simple program.  We will focus on commonly used software, e.g.,
the Firefox web browser, OpenOffice.org document editing suite, the
Pidgin instant messenger client, and a few other less-notable programs.

\section{Audience}
Unlike other visualizations, explanations, and visualization tools
which target the computer-savvy that just want to know more, we will
be targetting the average consumer.  The resulting visualization
should be visually appealing, interactive, rich in detail, yet still
clear.  The goal of the visualization is to break through some of the
abstractions to bring understanding, but at the same time should still
use these abstractions to focus the viewer's attention to important
aspects, while still providing detail that would be appreciated by
engineers.

\section{Methods}
We intend to use existing Linux profiling tools, namely
\texttt{valgrind} in \texttt{callgrind} mode. If necessary, we can
collect additional data with \texttt{OProfile} and \texttt{SystemTap}.

The resulting data will then be processed by a custom collection of
scripts, which will abstract the data and transform it into a format
that out web-based visualization toolkit can display.  These scripts
are already under development, and have produced functional
visualizations for Firefox and a Tetris game.

These scripts need to be further extended with context-sensitive
layout algorithms, and multi-dimensional layout that will show linkage
between different programs as they run on the same computer.

Additionally, some work must be performed to optimize the
visualization in Firefox---it is currently very resource intensive and
performs well only in Chrome.

\section{Data}
Data is obtained by running the desired program under
\texttt{valgrind}. \texttt{valgrind} was designed for use by software
engineers to ensure that their programs are behaving correctly, by
checking memory accesses, deadlocks, and keeping track of how many
times a particular function was called, and from where.
\texttt{valgrind} can also do additional checks to predict the
performance of a program on a particular processor, by analyzing cache
effects.

With such a powerful tool at our disposal, we are able to collect data
from any program without change, so long as debugging information is
available, limiting us to only open source software.  Fortunately,
many very popular programs are open source, so this is not a problem.

The data collection procedure entais:
\begin{enumerate}
\item Install \texttt{valgrind}
\item Install the desired program to be analyzed
\item Install the debug information for the desired program
\item Run the program under \texttt{valgrind}, e.g., \texttt{valgrind
    --tool=callgrind firefox}.
\end{enumerate}

\section{References}
The following references were surveyed in order to determine the
current state of visualization of the workings of a computer:

\begin{enumerate}
\item pfff, an ``API to write static analysis, dynamic analysis, code
  visualizations, code navigations, or style-preserving
  source-to-source transformations such as refactorings on source
  code''; \url{https://github.com/facebook/pfff/wiki/Examples}
\item ``Visualizing IPC Messages in Fennec'', which shows
  inter-procedure calls in Fennec, Mozilla's mobile Firefox browser;\\
  \url{http://mozakai.blogspot.com/2010/09/visualizing-ipc-messages-in-fennec.html}
\item ``Interactive Map of the Linux Kernel'';
  \url{http://www.makelinux.net/kernel_map/}
\item ``Visualizing dependencies of binaries and libraries in Linux'';\\
  \url{http://domseichter.blogspot.com/2008/02/visualize-dependencies-of-binaries-and.html}
\end{enumerate}

\end{document}
